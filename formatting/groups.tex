\section{Gruppierungen}
    Als Menschen assoziieren wir Sachen, die eng beieinander sind.
    Alle folgenden Prinzipien orientieren sich an diesem Grundsatz.
    Wir vergleichen dies wieder mit der Zeitung.
    Auch hier wird das Interresse des Lesers mehr durch die freien Flächen gelenkt, als durch den Text selbst.
    Es werden Artikel voneinander getrennt, zueinander in beziehung gesetzt und geschickt positioniert.
    Genau das sollten wir auch machen.

    \subsection{Positionierung}
        Die Reihenfolge spielt eine zentrale Rolle für die Lesbarkeit.
        \lstinputlisting[language=Java, caption={Positionierung Negativbeispiel \autocite{benson-2022}}, label=fig:positionierung]{test.java}
        Ein Negativbeispiel ist in \ref{fig:positionierung} gegeben.
        Hier muss beim Lesen gesprungen werden.
        Zunächst liegt der Fokus nicht auf den Einstiegspunkt \textit{a()} gelegt.
        Da wir von Oben nach Unten lesen, gehört diese Methode an den Anfang.
        Unter allen Umständen ist zu vermeiden, dass beim Lesen gesucht werden muss.\\
        Aber auch, wenn der Einstiegspunkt bekannt ist, so muss immernoch gesprungen werden.
        Von \textit{b()} nach \textit{c()}.
        Die Methoden sollten in der reihenfolge stehen, in der sie aufgerufen werden.
        Also $\rightarrow\textit{a()}\rightarrow\textit{b()}\rightarrow\textit{c()}$.
        Nur dann lassen sie sich in einer natürlichen Form lesen.\\
        Nun sind die wenigsten Programme so linear wie dieses.
        Hier sollte nach der semantischen Bedeutung bzw. der wichtigkeit, also wie stark die Methoden zusammenhängen, entschieden werden.\\
        Eine Ausnahme für dieses Konzept bilden Instanzvariablen in der Objekt Orientierten Programierung (im Folgenden OOP) oder globale Variablen in funktionalen Programiersprachen.
        Diese sollten einheitlich zu beginn der Klasse definiert werden.
        Dies ist dadurch zu begründen, dass davon auszugehen ist, dass diese die größte Kohäsion haben, also nach möglichkeit, in vielen Funktionen/Methoden verwendet werden.
        Außerdem weiß jeder sofort, wo er suchen muss.\\
        In Java gilt beispielhaft die Reihenfolge:
        \begin{enumerate}
            \item \textit{public static final} Variablen
            \item \textit{private static final} Variablen
            \item \textit{private static} Variablen
            \item \textit{private} Variablen
            \item \textit{public static} Methoden
            \item \textit{private static} Methoden
            \item \textit{public} Methoden
            \item \textit{private} Methoden
        \end{enumerate}
        Alles, was nicht genannt ist, sollte vermieden werden.

    \subsection{Vertikale Formatierung}
        Lehrzeilen sind das Mittel der Wahl für die vertikale Formatierung.
        Durch sie sind wir in der Lage, Abgrenzungen zu schaffen und die Stärke der Bindung zwischen verschiedenen Teilen auszudrücken.\\
        So ist es, zum Beispiel, wie auch in \ref{fig:positionierung} zu erkennen ist, üblich, zwischen verschiedenen Methoden eine Leerzeile zu lassen.
        Dadurch ist es einfacher zu erkennen, wann ein Scope aufhört.
        Dies bietet uns aber auch die Möglichkeit, zum Beispiel bei einem Getter und Setter, diese Lehrzeile weg zu lassen, um die zugehörigkeit zu verdeutlichen.
        Dies sollte allerdings nicht zu häufig passieren.
        Es gildt abzuwägen, was das Lesen im Einzelfall mehr erleichtert.\\
        Auf der anderen Seite können allerdings auch zwei Lerzeilen eine unabhängigkeit ausdrücken.
        Dies ist in einigen Programiersprachen zum Beispiel zwischen den Variablen und Methoden üblich.
