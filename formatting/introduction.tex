\section{Einleitung}
	So wie Softwareprojekte in den vergangenen Jahren immer weiter an Komplexität zugenommen haben, so rückt die Wartbarkeit immer weiter in den Fokus.
	Hierzu ist ist, neben den Verwendeten Tools, die "saubere" Strukturierung des Codes, der wichtigste Faktor.
	Es gibt viele Punkte zu beachten und noch mehr Ansätze und Methoden, um diese zu erreichen.
	Im Folgenden wollen wir uns mit der Formatierung, also der Verwendung von Whitespace zur besseren Lesbarkeit beschäftigen.
	\subsection{Wie lese ich eine Zeitung?}
		Wir waren doch eben noch in der Informatik, was hat eine Zeitung damit zu tun?\\
		Wenn wir von der "Newspaper Metaphor" sprechen, so einiges.
		Guter Code sollte sich genau so lesen, wie eine Zeitung.
		Jede Seite hat ein Thema, genau wie jede Klasse eine Aufgabe hat.
		Die Seite ist in Artikel gegliedert, die spannendsten dabei am auffälligsten und am weitesten oben, genau wie z.B. beim Scherenprinzip in Java die public static Variablen einer Klasse als erstes kommen, da ihnen, folgend aus der größten Emmergenz, die größte Bedeutung zukommen sollte.
		Whitespace unterteilt die Seite für uns.
		Und im Artikel natürlich von oben links nach unten rechts.\\
		Unsere Bemühungen beim Coden sollten immer dieser Darstellung gelten.
		Wir schreiben den Code nur ein mal, gelesen wird er aber unzählige male.\\
		Deshalb brauchen wir, als Entwickler, genau wie ein Journalist, viel training und vorzugsweise eine professionelle Ausbildung, um diese Darstellung, für Menschen ohne unser Wissen, so einfach wie möglich zu gestalten.\\
		Code ist dazu da, von Menschen gelesen zu werden, nicht von Maschinen.
		Deshalb macht es auch nur sinn, die Formatierung durch Whitespace, der von den aller meisten Programmiersprachen ignoriert wird, für uns lesbar zu gestalten.
		
