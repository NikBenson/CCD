\section{Fazit}
    Code wird einmal geschrieben und viele Male gelesen.
    Und zwar von Menschen, nicht von Maschinen.
    Auch wenn Maschinen den Code später interpretieren, so zielen wir darauf ab, wartbaren Code zu schreiben.\\
    Dafür ist die Formatierung besonders wichtig.
    Diese bietet uns Menschen Semantischen Wert, ohne an der Logik etwas zu ändern.\\
    Hierzu gibt es Stylerichtlinien.
    Diese sind teils mehr und teils weniger universell, aber niemals dogmatisch zu sehen.\\
    In vielen Fällen können Linter und Formatter hier heutzutage viel Arbeit abnehmen so wie den Style innerhalb eines Projektes vereinheitlichen.
    Dies gild meist der Lesbarkeit durch Whitespace.\\
    Darüber hinaus gild es, mit der Formatierung  Botschaften, meist in der Form von Zusammenhängen, zu vermitteln.
    Hierzu wird die Newspaper Metapher herangezogen: Zusammengehöriges sollte auch zusammen zu finden sein und umgekehrt.\\
    Dies ist allerdings immer die Aufgabe des Entwicklers.
    Beispielhaft ist hier die Reihenfolge von Variablen und Funktionen zu nennen.\\
    Zusätzlich spielt die Dichte/Offenheit und die Länge sowohl in der Vertikalen, als auch in der Horizontalen eine Wichtige Rolle.
    Dichte und Offenheit bedeutet dabei, Whitespace für den Abstand zu benutzen, um zusammenhänge zu verdeutlichen.
    Bei der Länge sollte versucht werden, analog zum Single Responsibility Principle, sowohl die Dateien, als auch die Zeilen nach Möglichkeit kurz zu halten.\\
    Abgesehen von diesen allgemeinen Regeln kann alles getan werden, was das Verständnis des Codes verbessert.
    Auch sollte sich an Styleguides von Programmiersprachen und dem eigenen Team gehalten werden.
