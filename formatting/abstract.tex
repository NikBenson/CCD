In der modernen Softwareentwicklung nimmt die Komplexität der Projekte, und damit auch die des Codes, immer mehr zu.
So haben die meisten Projekte verschiedene Abhängigkeiten auf andere Projekte oder sind, in großen Unternehmen, abhängig von Code aus anderen Abteilungen.\\
Der primäre Fokus darauf, laufzeiteffizienten und funktionierenden Code zu schreiben, ist deshalb in den allermeisten Fällen abgelöst worden, durch das Ziel, wartbaren Code zu schreiben.
Die Erkenntnis, dass Code sich heutzutage fortlaufend weiter entwickelt und niemals fertig ist, ist dabei von zentraler Wichtigkeit.\\
Auch wenn viele Methoden und Software (z.B. Git, Jira, etc.) den Prozess  begünstigen, so müssen wir auch neue Standards an den Code selbst stellen.\\
In der Standardliteratur \qq{Clean Code - A Handbook of Agile Software Craftmanship} stellt der Author Robert C. Martins in 17 Kapiteln seine wichtigsten Erkenntnisse hierzu dar.
Wir wollen uns, in dieser Arbeit, mit dem 5. befassen: Formatierung.
Unser Code sollte sich wie eine Zeitung lesen, aber wie schaffen wir das am besten?
Die Antwort beinhaltet mehr als nur kurze Dateien, an der richtigen Stelle kann eine Lehrzeile einen großen Unterschied machen.\\
Dies wollen wir uns im folgenden genauer anschauen.